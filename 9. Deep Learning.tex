\documentclass[11pt,aspectratio=169]{beamer}
\usetheme{Boadilla}
\usecolortheme{beaver}
\usepackage[utf8]{inputenc}
\usepackage[spanish,mexico]{babel}
\usepackage{amsmath}
\usepackage{amsfonts}
\usepackage{amssymb}
\usepackage{graphicx}
\usepackage{booktabs}
\usepackage{listings}
\usepackage{multicol}
\usepackage{multirow}
\usepackage{algorithm,algorithmic}
\author{Héctor Selley}
\title{Deep Learning}
\providecommand{\tightlist}{%
  \setlength{\itemsep}{0pt}\setlength{\parskip}{0pt}}
%\setbeamercovered{transparent} 
%\setbeamertemplate{navigation symbols}{} 
%\logo{} 
\institute{Universidad Anáhuac México} 
\date{\today} 
%\subject{} 
\begin{document}

\AtBeginSection[] % Do nothing for \section*
{
\begin{frame}<beamer>
\frametitle{Contenido}
\tableofcontents[currentsection]
\end{frame}
}

\begin{frame}
	\titlepage
\end{frame}

%\begin{frame}
%	\tableofcontents
%\end{frame}

\section{¿Qué es Deep Learning?}
\begin{frame}{¿Qué es Deep Learning?}
	\begin{itemize}
		\item El Deep Learning, también conocido como aprendizaje profundo, es una subrama del campo del aprendizaje automático (Machine Learning)\pause
	\end{itemize}
	\begin{figure}[H]
		\centering
		\includegraphics[scale=0.3]{../Programacion para ML/Contenido/img/AI-ML-DL.png} %img/AI-ML-DL.png
	\end{figure}
\end{frame}

\begin{frame}{¿Qué es Deep Learning?}
	\begin{itemize}
		\item Se enfoca en entrenar y utilizar redes neuronales artificiales profundas.\pause 
		\item Estas redes neuronales están compuestas por múltiples capas ocultas, lo que les permite aprender y representar de manera jerárquica 
			características y patrones cada vez más abstractos en los datos.\pause
		\item A diferencia de las redes neuronales tradicionales, que suelen tener una o dos capas ocultas, las redes neuronales profundas pueden 
			tener muchas capas ocultas\pause 
		\item Esto que les brinda una mayor capacidad de representación y un mayor poder de aprendizaje. \pause 
		\item Cada capa en una red neuronal profunda realiza transformaciones no lineales en los datos de entrada y pasa la información 
			a la siguiente capa.
	\end{itemize}
\end{frame}

\begin{frame}{¿Qué es Deep Learning?}
	\begin{itemize}
		\item El Deep Learning ha ganado popularidad y ha logrado avances significativos en áreas como: \pause
			\begin{itemize}
				\item Reconocimiento de imágenes\pause
				\item Procesamiento del lenguaje natural\pause
				\item Visión por computadora\pause
				\item Traducción automática\pause
				\item Entre otros. \pause
			\end{itemize} 
			
		\item Esto se debe a su capacidad para aprender automáticamente características y representaciones relevantes a partir de grandes 
			cantidades de datos, sin requerir una extracción manual de características.
	\end{itemize}
\end{frame}

\begin{frame}{¿Qué es Deep Learning?}
	\begin{itemize}
		\item Una de las razones clave detrás del éxito del Deep Learning es el uso de algoritmos de optimización y el desarrollo de 
			arquitecturas especializadas, como:\pause
			\begin{itemize}
				\item Redes neuronales convolucionales (CNN) para el procesamiento de imágenes \pause
				\item Redes neuronales recurrentes (RNN) para el procesamiento de secuencias.\pause 
			\end{itemize} 
		\item Estas arquitecturas se han diseñado específicamente para abordar los desafíos inherentes a diferentes tipos de datos 
			y han demostrado un rendimiento sobresaliente en muchas tareas de aprendizaje automático.
	\end{itemize}
\end{frame}

\begin{frame}{¿Qué es Deep Learning?}
	\begin{itemize}
		\item El Deep Learning ha impulsado avances significativos en áreas como: \pause 
			\begin{itemize}
				\item Visión artificial\pause 
				\item Procesamiento del lenguaje natural \pause
				\item Generación de imágenes y texto \pause
				\item Conducción autónoma \pause
				\item Medicina \pause 
				\item Biología \pause 
				\item Entre muchos otros campos.\pause 
			\end{itemize} 
		\item Su capacidad para aprender características y patrones complejos y realizar tareas sofisticadas ha llevado a su amplia 
			adopción en la comunidad científica y la industria.
	\end{itemize}
\end{frame}

\section{¿Cómo funciona el Deep Learning?}
\begin{frame}{¿Cómo funciona el Deep Learning?}
	\begin{itemize}
		\item El proceso de construcción del Deep Learning implica varias etapas clave.\pause 
		\item A continuación, se describen los pasos generales involucrados:
	\end{itemize}
\end{frame}

\begin{frame}{¿Cómo funciona el Deep Learning?}
	\begin{enumerate}
		\item \textbf{Definir el problema}:\pause Comienza identificando y definiendo claramente el problema que deseas resolver utilizando Deep Learning. \pause
			Esto implica comprender qué tipo de datos tienes, qué tipo de tarea deseas realizar (clasificación, regresión, generación, etc.) 
			y cuáles son los objetivos específicos del proyecto.\pause
		\item \textbf{Recopilar y preparar los datos}: Reúne los datos necesarios para entrenar y evaluar el modelo de Deep Learning.\pause 
			Estos datos deben ser representativos del problema que deseas resolver y deben estar debidamente etiquetados o anotados.\pause 
			Además, realiza una preparación de datos adecuada, que puede incluir la limpieza de datos, la normalización y la división en 
			conjuntos de entrenamiento, validación y prueba.\pause
	\end{enumerate}
\end{frame}

\begin{frame}{¿Cómo funciona el Deep Learning?}
	\begin{enumerate}
		\setcounter{enumi}{2}
		\item \textbf{Diseñar la arquitectura del modelo}: Selecciona la arquitectura apropiada para tu problema.\pause Esto implica determinar 
			el tipo de red neuronal (redes neuronales convolucionales, redes neuronales recurrentes, redes neuronales generativas, etc.) 
			y diseñar la estructura de capas, incluyendo el número de capas ocultas, la cantidad de neuronas en cada capa y la elección 
			de funciones de activación.\pause
		\item \textbf{Inicializar el modelo}: Inicializa los parámetros del modelo, como los pesos de las conexiones, de manera aleatoria 
			o utilizando algún enfoque de inicialización específico.\pause Esto establece el punto de partida para el proceso de entrenamiento.
	\end{enumerate}
\end{frame}


\begin{frame}{¿Cómo funciona el Deep Learning?}
	\begin{enumerate}
		\setcounter{enumi}{4}
		\item \textbf{Entrenar el modelo}: Utiliza los datos de entrenamiento para ajustar los pesos y los parámetros del modelo.\pause 
			Esto implica alimentar los datos a la red neuronal, realizar la propagación hacia adelante (forward propagation) para calcular las salidas, 
			calcular la pérdida (loss) o el error, y luego realizar la propagación hacia atrás (backpropagation) 
			para actualizar los pesos a través de algoritmos de optimización, como el descenso del gradiente, con el objetivo de minimizar la pérdida.
	\end{enumerate}
\end{frame}

\begin{frame}{¿Cómo funciona el Deep Learning?}
	\begin{enumerate}
		\setcounter{enumi}{5}
		\item \textbf{Ajustar hiperparámetros}: Los hiperparámetros, como la tasa de aprendizaje, el tamaño del lote (batch size), el número de 
			épocas, etc., afectan el rendimiento del modelo.\pause Realiza experimentos y ajusta estos hiperparámetros para optimizar el rendimiento 
			del modelo en el conjunto de datos de validación.\pause
		\item \textbf{Evaluar y afinar el modelo}: Evalúa el rendimiento del modelo utilizando el conjunto de datos de prueba o datos no vistos.\pause 
			Analiza métricas de rendimiento relevantes y realiza ajustes adicionales en la arquitectura del modelo o en los hiperparámetros según 
			sea necesario.
	\end{enumerate}
\end{frame}

\begin{frame}{¿Cómo funciona el Deep Learning?}
	\begin{enumerate}
		\setcounter{enumi}{7}
		\item \textbf{Despliegue y predicción}: Una vez que estás satisfecho con el rendimiento del modelo, puedes implementarlo en producción y 
			utilizarlo para realizar predicciones en datos nuevos y no vistos.\pause
	\end{enumerate}
	\begin{itemize}
		\item Es importante tener en cuenta que el proceso de construcción del Deep Learning es iterativo y requiere experimentación, ajustes 
			y refinamientos para lograr un modelo óptimo. \pause
		\item Además, también es fundamental contar con un conjunto de datos de calidad y suficiente capacidad computacional para entrenar 
			modelos de Deep Learning de manera eficiente.
	\end{itemize}
\end{frame}

\begin{frame}{¿Cómo funciona el Deep Learning?}
	
\end{frame}
















\end{document}
