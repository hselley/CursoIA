\documentclass[11pt,aspectratio=169]{beamer}
\usetheme{Boadilla}
\usecolortheme{beaver}
\usepackage[utf8]{inputenc}
\usepackage[spanish,mexico]{babel}
\usepackage{amsmath}
\usepackage{amsfonts}
\usepackage{amssymb}
\usepackage{graphicx}
\usepackage{booktabs}
\usepackage{listings}
\usepackage{multicol}
\usepackage{multirow}
\usepackage{algorithm,algorithmic}
\author{Héctor Selley}
\title{Deep Learning}
\providecommand{\tightlist}{%
  \setlength{\itemsep}{0pt}\setlength{\parskip}{0pt}}
%\setbeamercovered{transparent} 
%\setbeamertemplate{navigation symbols}{} 
%\logo{} 
\institute{Universidad Anáhuac México} 
\date{\today} 
%\subject{} 
\begin{document}

\AtBeginSection[] % Do nothing for \section*
{
\begin{frame}<beamer>
\frametitle{Contenido}
\tableofcontents[currentsection]
\end{frame}
}

\begin{frame}
	\titlepage
\end{frame}

%\begin{frame}
%	\tableofcontents
%\end{frame}

\section{Deep Learning}
\begin{frame}{Deep Learning}
\begin{itemize}
	\item Es un método específico de machine learning que incorpora las redes neuronales en capas sucesivas para aprender de los datos de
		manera iterativa.\pause
	\item El deep learning es especialmente útil cuando se trata de aprender patrones de datos no estructurados.\pause
	\item Las redes neuronales complejas de deep learning están diseñadas para emular cómo funciona el cerebro humano, así que las computadoras 
		pueden ser entrenadas para lidiar con abstracciones y problemas mal definidos.\pause
	\item Las redes neuronales y el deep learning se utilizan a menudo en el reconocimiento de imágenes, voz y aplicaciones de visión de computadora.
\end{itemize}
\end{frame}

\begin{frame}{Deep Learning}
\begin{itemize}
	\item Las redes neuronales profundas tienen un enorme potencial para aprender funciones, patrones y representaciones no lineales complejas.\pause
	\item Su poder está impulsando la investigación en múltiples campos, incluida la visión por computadora, el análisis audiovisual, los chatbots 
		y la comprensión del lenguaje natural, por nombrar algunos.
\end{itemize}
\end{frame}

\begin{frame}{Deep Learning}
\begin{figure}[H]
	\centering
	\includegraphics[scale=0.35]{../Programacion para ML/Contenido/img/AI-ML-DL.png} %img/AI-ML-DL.png
\end{figure}
\end{frame}

\end{document}
