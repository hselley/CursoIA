\documentclass[11pt,aspectratio=169]{beamer}
\usetheme{Boadilla}
\usecolortheme{beaver}
\usepackage[utf8]{inputenc}
\usepackage[spanish,mexico]{babel}
\usepackage{amsmath}
\usepackage{amsfonts}
\usepackage{amssymb}
\usepackage{graphicx}
\usepackage{booktabs}
\usepackage{listings}
\usepackage{multicol}
\usepackage{multirow}
\usepackage{algorithm,algorithmic}
\author{Héctor Selley}
\title{Tendencias de la IA}
\providecommand{\tightlist}{%
  \setlength{\itemsep}{0pt}\setlength{\parskip}{0pt}}
%\setbeamercovered{transparent} 
%\setbeamertemplate{navigation symbols}{} 
%\logo{} 
\institute{Universidad Anáhuac México} 
\date{\today} 
%\subject{} 
\begin{document}

\AtBeginSection[] % Do nothing for \section*
{
\begin{frame}<beamer>
\frametitle{Contenido}
\tableofcontents[currentsection]
\end{frame}
}

\begin{frame}
	\titlepage
\end{frame}

%\begin{frame}
%	\tableofcontents
%\end{frame}

\section{Tendencias}
\begin{frame}{Tendencias}
	Algunas de las tendencias de la IA hasta Septiembre del 2021 son:
	\begin{itemize}
		\item \textbf{Aprendizaje profundo (Deep Learning)}: El aprendizaje profundo ha sido una tendencia dominante en la IA 
			en los últimos años y es probable que continúe siéndolo. Los modelos de aprendizaje profundo, como las redes 
			neuronales profundas, han demostrado un gran éxito en tareas como el reconocimiento de imágenes, el procesamiento 
			del lenguaje natural y la generación de contenido.\pause
		\item \textbf{IA explicable y ética}: Con el crecimiento de la IA, ha surgido una mayor preocupación por la transparencia, 
			la ética y la explicabilidad de los sistemas de IA. Existe una creciente demanda de algoritmos y modelos que sean 
			comprensibles y que puedan explicar el razonamiento detrás de sus decisiones.
	\end{itemize}
\end{frame}

\begin{frame}{Tendencias}
	\begin{itemize}
		\item \textbf{Inteligencia artificial en la atención médica}: La IA tiene un gran potencial para revolucionar la atención 
			médica. Se espera que se utilice para el diagnóstico médico, la interpretación de imágenes médicas, el descubrimiento 
			de medicamentos y la atención personalizada. La capacidad de los modelos de IA para analizar grandes cantidades de datos 
			y encontrar patrones puede mejorar significativamente la precisión y la eficiencia de los diagnósticos y tratamientos 
			médicos.\pause
		\item \textbf{IA en vehículos autónomos}: Los vehículos autónomos continúan siendo una tendencia en el campo de la IA 
			y la industria automotriz. La IA se utiliza para desarrollar sistemas de detección, reconocimiento y toma de decisiones 
			en vehículos sin conductor, con el objetivo de mejorar la seguridad vial y la eficiencia del transporte.
	\end{itemize}
\end{frame}

\begin{frame}{Tendencias}
	\begin{itemize}
		\item \textbf{IA en el procesamiento del lenguaje natural (NLP)}: El procesamiento del lenguaje natural es un área de rápido 
			crecimiento en la IA. Se espera que los modelos de lenguaje generativos, como los basados en GPT, mejoren en su capacidad 
			para entender y generar texto de manera más coherente y contextualmente relevante.\pause
		\item \textbf{Robótica y automatización}: La IA se utiliza cada vez más en aplicaciones robóticas y de automatización. 
			Los robots y sistemas automatizados están siendo diseñados para realizar tareas complejas y adaptativas en entornos 
			diversos, lo que puede aumentar la eficiencia y la productividad en una amplia gama de industrias.
	\end{itemize}
\end{frame}

\begin{frame}{Tendencias}
	\begin{itemize}
		\item \textbf{Privacidad y seguridad en la IA}: Con el aumento en la cantidad de datos utilizados en aplicaciones de IA, 
			la privacidad y la seguridad de esos datos se han convertido en preocupaciones importantes. Se espera que se desarrollen 
			nuevas técnicas de privacidad y métodos de protección de datos para abordar estas preocupaciones.\pause
	\end{itemize}
	El campo de la IA es altamente dinámico y en constante evolución, por lo que es indispensable mantenerse actualizado de las nuevas
	aplicaciones.
\end{frame}

\section{ChatGPT}
\begin{frame}{Caso especial: ChatGPT}
	\begin{block}{ChatGPT}
		Es un modelo de lenguaje basado en la arquitectura GPT (Generative Pre-trained Transformer), desarrollado por 
		OpenAI\footnotemark. Es un 
		modelo de aprendizaje automático que utiliza redes neuronales para generar respuestas a partir de las entradas de texto que recibe.	
	\end{block}
	\footnotetext{\url{https://chat.openai.com/}}
\end{frame}

\begin{frame}{¿Cómo funciona?}
	ChatGPT esta compuesto por cuatro grandes módulos.
	\begin{enumerate}
		\item Arquitectura Transformer\pause
		\item Entrenamiento pre-entrenado\pause
		\item Afinamiento específico\pause
		\item Generación de respuestas
	\end{enumerate}
\end{frame}

\begin{frame}{Arquitectura Transformer}
	\textbf{Arquitectura Transformer}
	\begin{itemize}
		\item ChatGPT se basa en la arquitectura Transformer, que es una red neuronal de aprendizaje 
			profundo diseñada para procesar secuencias de texto. \pause
		\item Esta arquitectura se compone de múltiples capas de atención y feed-forward, \pause
		\item lo que permite capturar las relaciones entre las palabras y comprender el contexto en 
			el que se encuentra cada palabra.
	\end{itemize}
\end{frame}

\begin{frame}{Entrenamiento pre-entrenado}
	\textbf{Entrenamiento pre-entrenado}
	\begin{itemize}
		\item ChatGPT es entrenado utilizando un proceso de pre-entrenamiento y afinamiento.\pause 
		\item En el pre-entrenamiento, se alimenta al modelo grandes cantidades de texto de diversas fuentes de Internet.\pause 
		\item El modelo trata de predecir la siguiente palabra en una oración dada la información anterior. \pause 
		Este proceso permite que el modelo aprenda patrones y estructuras lingüísticas.
	\end{itemize}
\end{frame}

\begin{frame}{Afinamiento específico}
	\textbf{Afinamiento específico}
	\begin{itemize}
		\item Después del pre-entrenamiento, el modelo se ajusta o "afina" utilizando datos de entrenamiento 
			específicos para la tarea deseada.\pause
		\item En el caso de ChatGPT, se utiliza un conjunto de datos donde se muestra una conversación 
			y las respuestas correspondientes. \pause 
		\item Durante el afinamiento, el modelo aprende a generar respuestas relevantes y coherentes 
			basadas en el contexto.\pause
	\end{itemize}
\end{frame}

\begin{frame}{Generación de respuestas}
	\textbf{Generación de respuestas}
	\begin{itemize}
		\item Una vez que ChatGPT está entrenado, se puede utilizar para generar respuestas a partir de
			las entradas de texto proporcionadas.\pause 
		\item Cuando se le presenta un mensaje, el modelo procesa el texto y lo utiliza como contexto 
			para generar una respuesta adecuada. \pause 
		\item La respuesta se genera de manera probabilística, tomando en cuenta el contexto anterior 
			y el conocimiento adquirido durante el entrenamiento.
	\end{itemize}
\end{frame}	

\end{document}
