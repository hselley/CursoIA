\documentclass[11pt,aspectratio=169]{beamer}
\usetheme{Boadilla}
\usecolortheme{beaver}
\usepackage[utf8]{inputenc}
\usepackage[spanish,mexico]{babel}
\usepackage{amsmath}
\usepackage{amsfonts}
\usepackage{amssymb}
\usepackage{graphicx}
\usepackage{booktabs}
\usepackage{listings}
\usepackage{multicol}
\usepackage{multirow}
\usepackage{algorithm,algorithmic}
\author{Héctor Selley}
\title{Árboles de decisión}
\providecommand{\tightlist}{%
  \setlength{\itemsep}{0pt}\setlength{\parskip}{0pt}}
%\setbeamercovered{transparent} 
%\setbeamertemplate{navigation symbols}{} 
%\logo{} 
\institute{Universidad Anáhuac México} 
\date{\today} 
%\subject{} 
\begin{document}

\AtBeginSection[] % Do nothing for \section*
{
\begin{frame}<beamer>
\frametitle{Contenido}
\tableofcontents[currentsection]
\end{frame}
}

\begin{frame}
	\titlepage
\end{frame}

%\begin{frame}
%	\tableofcontents
%\end{frame}

\section{¿Qué es un árbol de decisión?}
\begin{frame}{¿Qué es un árbol de decisión?}
	\begin{itemize}\pause
		\item Un árbol de decisión es un modelo de predicción. \pause
		\item Se utiliza en diversas disciplinas como la Inteligencia Artificial, 
		Medicina, Ingeniería, Ciencia de Datos y la Economía, entre muchas otras. \pause
		\item Los árboles se construyen desde un conjunto de datos \pause 
		\item Los diagramas resultantes son similares a los sistemas de predicción que se basan
		en reglas \pause
		\item Sirven para categorizar una serie de condiciones que ocurren en forma sucesiva.
	\end{itemize}
\end{frame}

\begin{frame}{¿Qué es un árbol de decisión?}
	Los árboles de decisión se utilizan en cualquier proceso que implique una toma de decisión\pause, por ejemplo:\pause
	\begin{itemize}
		\item Búsqueda binaria \pause
		\item Sistemas expertos \pause
		\item Árboles de juego
	\end{itemize}	
\end{frame}

\begin{frame}{¿Qué es un árbol de decisión?}
	\begin{itemize}
		\item Los árboles de decisión son generalmente binarios \pause
		\item Significa que pueden tomar dos opciones \pause
		\item Aunque es posible que existan árboles de tres o más opciones.
	\end{itemize}
\end{frame}

\begin{frame}{¿Para qué sirve un árbol de decisión?}
	Objetivos del árbol de decisión: \pause
	\begin{itemize}
		\item Encontrar un árbol binario que clasifique datos de entrada 
			con una \textit{dispersión} mínima. \pause
		\item Calcular la eficiencia del proceso de clasificación mediante 
			la \textit{dispersión}.
	\end{itemize}	
\end{frame}

\begin{frame}{¿Qué es un árbol de decisión?}
	\begin{block}{Árbol de decisión}
		Árbol de decisión es una técnica de estructura de datos jerárquicos 
		que se utiliza para la clasificación y regresión de datos.\pause 
		Este método emplea la técnica \textit{divide y vencerás}, mediante 
		la cual encuentra recursivamente la separación por clasificación de 
		los datos de entrada. 
	\end{block}
\end{frame}

\begin{frame}{¿Qué es un árbol de decisión?}
	\begin{itemize}
		\item Un árbol de decisión es un grafo que consiste en nodos y aristas.\pause
		\item Cada nodo puede tener máximo dos aristas, razón por lo que se le denomina 
		como binario.\pause
		\item Un árbol de decisión responde una pregunta acerca de los datos y los 
		clasifica de acuerdo con la respuesta de dicha pregunta. \pause
	\end{itemize}

	Utilizaremos algunos ejemplos para explicar los árboles de decisión, cómo se 
	definen y construyen.
\end{frame}

\section{Ejemplos}
\subsection{Ejemplo 1}
\begin{frame}{Ejemplo 1}
	La figura \ref{fig:arbol1} muestra un árbol de decisión que mediante una pregunta, 
	cuya respuesta puede ser verdadero o falso, clasifica los datos de entrada en dos 
	grupos. \pause

	\begin{figure}[H]
		\centering
		\includegraphics[scale=0.4]{../../Libro ML/Decision Tree/notas/img/arbol1.pdf}
		\caption{Ejemplo de árbol de decisión.}
		\label{fig:arbol1}
	\end{figure}
\end{frame}

\begin{frame}{Ejemplo 1}
	\begin{itemize}
		\item En los árboles, los \textbf{nodos} se representan con círculos o elipses 
		en los cuales se aloja una pregunta \pause 
		\item Las aristas son la conexión entre ellos a través de la respuesta de la 
		pregunta. 
		\item Se denomina \textbf{rama} al conjunto de al menos dos nodos conectados 
		por una arista.
	\end{itemize}
\end{frame}

\begin{frame}{Ejemplo 1}
	\begin{itemize}
		\item Imagine que tiene un conjunto de datos que desea clasificar mediante 
		una pregunta cuya respuesta es verdadero o falso. (figura \ref{fig:arbol1}) \pause
		\item Esto permite clasificar los datos en dos grupos, uno cuya respuesta 
		fue verdadera y otro cuya respuesta fue falsa. \pause
		\item Para un árbol tan pequeño como el de este ejemplo, la separación de 
		los datos es muy limitada por lo que se busca mejorarla empleando más nodos 
		en el árbol, lo que significa un mayor número de categorías.\pause
		\item Adicionalmente, la pregunta sólo acepta respuestas absolutas, si se 
		requiere de un rango de respuestas, por ejemplo, un rango de números 
		habría que modificar el árbol.
	\end{itemize}
\end{frame}

\subsection{Ejemplo 2}
\begin{frame}{Ejemplo 2}
	\begin{block}{Descripción del problema}
		Construyamos un árbol con más nodos y ramificaciones, para este ejemplo se clasifica una persona de acuerdo con su edad. \pause
		Se clasifica a una persona como adulto si su edad es mayor o igual a 18 años, como adolescente si está entre los 12 y 18 años, 
		como niño si está entre los 2 y 12 años y como bebé si es menor a 2 años. \pause
		El árbol resultante se muestra en la figura \ref{fig:arbol2}.
	\end{block}
\end{frame}

\begin{frame}{Ejemplo 2}
	\begin{figure}[H]
		\centering
		\includegraphics[scale=0.3]{../../Libro ML/Decision Tree/notas/img/arbol2.pdf}
		\caption{Árbol de decisión con más nodos.}
		\label{fig:arbol2}
	\end{figure}
\end{frame}

\begin{frame}{Ejemplo 2}
	\begin{itemize}
		\item En el árbol resultante de la figura \ref{fig:arbol2} clasifica a las personas de acuerdo con su edad 
			utilizando el criterio antes mencionado. \pause
		\item En el árbol las personas han sido clasificadas en los nodos: adulto, adolescente, niño y bebé.\pause
		\item A los nodos que tienen flechas que llegan a él pero no salen de él, se les denomina como 
			\textbf{nodos terminales} o de decisión.\pause
		\item Al nodo inicial del que sólo salen flechas de el pero no entran, se le denomina \textbf{nodo raíz} 
			o simplemente \textbf{raíz}.\pause
		\item Los demás simplemente se les denomina como \textbf{nodos}. 
	\end{itemize}
\end{frame}

\begin{frame}{Ejemplo 2}
	\begin{itemize}
		\item En la figura \ref{fig:arbol2} el nodo en rojo es la raíz, los nodos en verde son terminales y los 
			azules son simplemente nodos.\pause
		\item En el árbol de decisión de la figura \ref{fig:arbol2} clasifica adecuadamente a las personas, 
			dado que una persona sólo tiene una edad, la clasificación es perfecta de esa forma.\pause
		\item Imagínese que deseamos clasificar personas de acuerdo con otro criterio, un criterio en el cual 
			la respuesta no será tan específica como la edad o incluso puede que no haya una respuesta.\pause
		\item Por ejemplo, imagine que deseamos clasificar personas de acuerdo con su sabor preferido de helado,
			puede que tenga uno, varios o incluso ninguno.\pause
		\item En una situación como ésta, habrá una \textbf{impureza} en la clasificación. 
	\end{itemize}
\end{frame}

\subsection{Ejemplo 3}
\begin{frame}{Ejemplo 3}
	\begin{block}{Descripción del problema}
		Supongamos que a través de un estudio se obtiene un conjunto de datos acerca de 303 pacientes en los 
		que se sabe si sufren de dolor en el pecho, tienen buena circulación sanguínea, arterias bloqueadas 
		y ataque cardíaco.\pause Se sabe que existe una relación entre estos padecimientos, pero se busca 
		clasificar a los pacientes de la mejor forma posible.\pause Se desea determinar las causas que 
		provocan un ataque cardíaco y clasificar a los pacientes de acuerdo con ello, además un paciente que
		sufre un ataque cardíaco no presenta necesariamente todos los síntomas. \pause 
	\end{block}
\end{frame}

\begin{frame}{Ejemplo 3.}
	
	\begin{table}
		\centering
		\begin{tabular}{c|c|c|c}
			\toprule
			Dolor de pecho & Buena circulación & Arterias bloqueadas & Ataque cardíaco \\
			\midrule		
			No & No & No & No \\\midrule
			Si & Si & Si & Si \\\midrule
			-- & Si & No & No \\\midrule
			Si & Si & No & No \\\midrule
			Si & No & -- & Si \\\midrule
			No & -- & Si & Si \\\midrule
			\vdots & \vdots & \vdots & \vdots \\
			\bottomrule
		\end{tabular}
		\caption{Resultados del estudio de cada uno de los pacientes.}
		\label{tab:datosPacientes}
	\end{table}
\end{frame}

\begin{frame}{Ejemplo 3}
	\begin{itemize}
		\item La tabla \ref{tab:datosTotalesPacientes} muestra la cantidad total de pacientes obtenidos 
			por categoría a través del estudio. \pause 
		\item Observe que el total de pacientes por síntoma no es igual para todas las categorías \pause
		\item Esto es debido a que no se sabe la información completa para todos los pacientes. \pause
	\end{itemize}

\begin{table}
	\centering
	\begin{tabular}{c|c|c|c|c}
		\toprule
		 & Dolor de pecho & Buena circulación & Arterias bloqueadas & Ataque cardíaco \\
		\midrule		
		Si & 144 & 164 & 123 & 137 \\\midrule
		No & 159 & 133 & 174 & 160 \\\midrule
		Total & 303 & 297 & 297 & 297\\
		\bottomrule
	\end{tabular}
	\caption{Resultados totales del estudio de los pacientes.}
	\label{tab:datosTotalesPacientes}
\end{table}
\end{frame}

\begin{frame}{Ejemplo 3}
	\begin{itemize}
		\item Dado que se desea clasificar a los 303 pacientes de acuerdo con el síntoma que les ocasionó un infarto \pause
		\item Se necesita determinar mediante cuál de los tres síntomas se debe clasificar en primer lugar.\pause
		\item Este primer síntoma con el que se comience la clasificación se convertirá en el \textbf{nodo raíz} del árbol.\pause
		\item Por esta razón se analizará cuál de los tres síntomas separa mejor a los pacientes que sufrieron un ataque cardíaco.
	\end{itemize}
\end{frame}

\begin{frame}{Ejemplo 3}
	\begin{itemize}
		\item En la tabla \ref{tab:datosTotalesDesglosados} se muestran los datos totales desglosados por síntoma y si sufrieron o
			no un ataque cardíaco. \pause
		\item Utilizando estos datos se puede llevar a cabo la separación de los pacientes respecto a si sufren de dolor de pecho 
			o no. \pause
		\item La separación se muestra en la figura \ref{fig:arbol3-1}.
	\end{itemize}
\end{frame}

\begin{frame}{Ejemplo 3}
	\begin{table}
		\centering
		\begin{tabular}{c||cc|cc|cc}
		  \toprule
		  \multirow{2}{*}{Ataque} & \multicolumn{2}{c}{Dolor de pecho} & \multicolumn{2}{c}{Buena circulación} & \multicolumn{2}{c}{Arterias bloqueadas} \\  
								  & Si               & No              & Si                & No                & Si                 & No                 \\ \midrule
		  Si                      & 105              & 34              & 37                & 33                & 92                 & 45                 \\ \midrule
		  No                      & 39               & 125             & 127               & 100               & 31                 & 129                \\ \midrule\midrule
		  \textbf{Total}          & 144              & 159             & 164               & 133               & 123                & 174                \\ 
		  \bottomrule
		\end{tabular}
		\caption{Resultados Desglosados de los pacientes.}
		\label{tab:datosTotalesDesglosados}
	  \end{table}
\end{frame}

\begin{frame}{Ejemplo 3}
	\begin{figure}[H]
		\centering
		\includegraphics[scale=0.7]{../../Libro ML/Decision Tree/notas/img/arbol3-1.pdf}
		\caption{Separación mediante dolor de pecho.}
		\label{fig:arbol3-1}
	\end{figure}
\end{frame}

\begin{frame}{Ejemplo 3}
	\begin{itemize}
		\item Observe que la separación no es perfecta, dado que en cada rama se encuentran pacientes que han 
			sufrido un ataque y otros que no.\pause
		\item Esto es lo que se denomina como \textbf{impureza}. \pause
		\item Resulta intuitivo buscar una separación que ocasione una menor impureza\pause
		\item Por lo que para medirla resulta indispensable una métrica. \pause
		\item Para medir la impureza se utiliza el
		\textbf{índice de impureza de Gini}\cite{breiman2017classification}\cite{gelfand}, mediante la expresión (\ref{eq:gini}). 
	\end{itemize} 
\end{frame}

\begin{frame}{Ejemplo 3}
	\begin{equation}
		G = 1 - (\mbox{Probabilidad Si})^2 - (\mbox{Probabilidad No})^2 
		\label{eq:gini}
	\end{equation}
	\begin{itemize}
		\item Donde $G$ representa el índice de impureza Gini \pause
		\item \textit{Probabilidad Si} y \textit{Probabilidad No} son la probabilidad de que en un paciente tenga 
			o no tenga dolor de pecho respectivamente.\pause
		\item La probabilidad simplemente se calcula mediante el cociente de los pacientes con o sin dolor entre 
			el total de pacientes.\pause
		\item De esta forma, se calcula el índice para cada separación posible y se elige aquella que tenga el 
			menor valor de impureza. 
	\end{itemize}	
\end{frame}

\begin{frame}{Ejemplo 3}
	\begin{itemize}
		\item Para calcular el índice de impureza para la separación con respecto a dolor de pecho $G_{DP}$, 
		se debe analizar a su vez el índice para cada una de las ramas de dicha separación\pause
		\item Esto es $G_{Si}$ y $G_{No}$.\pause
		\item Por lo tanto, el cálculo de la impureza para cada caso es el siguiente:
	\end{itemize}
\end{frame}

\begin{frame}{Ejemplo 3}
	\begin{align*}
		G_{Si} &= 1 - (\mbox{Probabilidad Si})^2 - (\mbox{Probabilidad No})^2  \\
			&= 1-\left(\dfrac{105}{105+39}\right)^2 -\left(\dfrac{39}{105+39} \right)^2\\ 
			&= 0.3949 
	\end{align*}\pause
	\begin{align*}
		G_{No} &= 1- (\mbox{Probabilidad Si})^2 - (\mbox{Probabilidad No})^2  \\
			&= 1-\left(\dfrac{34}{34+125}\right)^2 -\left(\dfrac{125}{34+125} \right)^2\\
			&= 0.3362
	\end{align*}
\end{frame}

\begin{frame}{Ejemplo 3}
	\begin{itemize}
		\item Una vez calculado el índice de impureza de Gini para las dos hojas terminales, 
			se calcula el índice total de impureza al separar los pacientes mediante el dolor de pecho.\pause
		\item Sin embargo, dado que ambas hojas no representan la misma cantidad de pacientes se necesita 
			utilizar el promedio ponderado de los índices de impureza para cada rama.\pause 
		\item Esto está dado en la expresión (\ref{eq:promedioPonderado}).
	\end{itemize}
\end{frame}

\begin{frame}{Ejemplo 3}
	\begin{equation}
		G = \dfrac{P_{Si}}{P_{Si} + P_{No}}G_{Si} + \dfrac{P_{No}}{P_{Si}+P_{No}}G_{No}
		\label{eq:promedioPonderado}
	\end{equation}
	\begin{itemize}
		\item Donde $P_{Si}$ es la cantidad total de pacientes que sufren dolor de pecho y $P_{No}$ la cantidad de pacientes que no lo sufren. 
	\end{itemize}
\end{frame}

\begin{frame}{Ejemplo 3}
	De esta forma, el índice total de impureza al separar pacientes mediante dolor de pecho $G_{DP}$ es:\pause
	\begin{align*}
		G_{DP} &= \dfrac{P_{Si}}{P_{Si} + P_{No}}G_{Si} + \dfrac{P_{No}}{P_{Si}+P_{No}}G_{No} \\
			&= \dfrac{105+39}{105+39+34+125}\times (0.3949) + \dfrac{34+125}{105+39+34+125}\times (0.3362) \\
			&= 0.3641
	\end{align*}		
\end{frame}

\begin{frame}{Ejemplo 3}
	\begin{itemize}
		\item Ahora se realiza la separación respecto a la buena circulación de la sangre y todos los cálculos 
		correspondientes para obtener el índice de impureza $G_{BC}$ para la separación mediante buena 
		circulación de la sangre.
	\end{itemize}\pause

\begin{figure}[H]
	\centering
	\includegraphics[scale=0.45]{../../Libro ML/Decision Tree/notas/img/arbol3-2.pdf}
	\caption{Separación mediante buena circulación de la sangre.}
	\label{fig:arbol3-2}
\end{figure}
\end{frame}

\begin{frame}{Ejemplo 3}
	\begin{align*}
		G_{Si} &= 1 - (\mbox{Probabilidad Si})^2 - (\mbox{Probabilidad No})^2  \\
			&= 1-\left(\dfrac{37}{37+127}\right)^2 -\left(\dfrac{127}{37+127} \right)^2\\
			&= 0.3494
	\end{align*}\pause
	\begin{align*}
		G_{No} &= 1- (\mbox{Probabilidad Si})^2 - (\mbox{Probabilidad No})^2  \\
			&= 1-\left(\dfrac{100}{100+33}\right)^2 -\left(\dfrac{33}{100+33} \right)^2\\
			&= 0.3731
	\end{align*}
\end{frame}

\begin{frame}{Ejemplo 3}
	\begin{align*}
		G_{BC} &= \dfrac{P_{Si}}{P_{Si} + P_{No}}G_{Si} + \dfrac{P_{No}}{P_{Si}+P_{No}}G_{No} \\
			&= \dfrac{37+127}{37+127+100+33}\times (0.3494) + \dfrac{100+33}{37+127+100+33}\times (0.3731) \\
			&= 0.3600
	\end{align*}
\end{frame}

\begin{frame}{Ejemplo 3}
	\begin{itemize}
		\item Por último, se realiza la separación mediante las arterias bloqueadas y los cálculos 
		correspondientes para obtener el índice de impureza $G_{AB}$ para la separación mediante las 
		arterias bloqueadas.
	\end{itemize}\pause
	
	\begin{figure}[H]
		\centering
		\includegraphics[scale=0.45]{../../Libro ML/Decision Tree/notas/img/arbol3-3.pdf}
		\caption{Separación mediante arterias bloqueadas.}
		\label{fig:arbol3-3}
	\end{figure}
\end{frame}

\begin{frame}{Ejemplo 3}
	\begin{align*}
		G_{Si} &= 1 - (\mbox{Probabilidad Si})^2 - (\mbox{Probabilidad No})^2  \\
			&= 1-\left(\dfrac{92}{92+31}\right)^2 - \left(\dfrac{31}{92+31} \right)^2\\
			&= 0.3770
	\end{align*}\pause
	\begin{align*}
		G_{No} &= 1 - (\mbox{Probabilidad Si})^2 - (\mbox{Probabilidad No})^2  \\
			&= 1-\left(\dfrac{45}{45+129}\right)^2 - \left(\dfrac{129}{45+129}\right)^2\\
			&= 0.3834
	\end{align*}
\end{frame}

\begin{frame}{Ejemplo 3}
	\begin{align*}
		G_{AB} &= \dfrac{P_{Si}}{P_{Si} + P_{No}}G_{Si} + \dfrac{P_{No}}{P_{Si}+P_{No}}G_{No} \\
			&= \dfrac{92+31}{92+31+45+129}\times (0.3770) + \dfrac{45+129}{92+31+45+129}\times (0.3834)\\
			&= 0.3808
	\end{align*}
\end{frame}

\begin{frame}{Ejemplo 3}
	\begin{itemize}
		\item La impureza es un efecto no deseado en la separación de pacientes bajo cualquier criterio\pause
		\item Por esa razón es que se decidirá entre las posibles separaciones por aquella que tenga un 
			menor valor de impureza.\pause
		\item Comparando los valores de impureza:\pause
			\begin{itemize}
				\item $G_{DP} = 0.3641$
				\item $G_{BC} = 0.36$
				\item $G_{AB} = 0.3808$
			\end{itemize}
		\item Se decide por $G_{BC}$ debido a que su valor es el menor de todos.\pause
		\item Esto significa que buena circulación se convertirá en el primer nodo, el \textbf{nodo raíz}.\pause
		\item El árbol hasta ahora queda como en la figura \ref{fig:arbol3-2}.
	\end{itemize}
\end{frame}

\begin{frame}{Ejemplo 3}
	\begin{figure}[H]
		\centering
		\includegraphics[scale=0.6]{../../Libro ML/Decision Tree/notas/img/arbol3-2.pdf}
		\caption{Separación mediante buena circulación de la sangre.}
	\end{figure}
\end{frame}

\begin{frame}{Ejemplo 3}
	\begin{itemize}
		\item Ahora debemos separar los pacientes respecto a dolor de pecho o arterias 
			bloqueadas para cada hoja.\pause  
		\item En primer lugar, se analiza la rama verdadera (izquierda) del árbol de la figura 
			\ref{fig:arbol3-2}.\pause
		\item Observe que el número total de pacientes de esa rama es 164, de los cuales 37 
			sufrieron ataque y 127 no lo sufrieron.
		\begin{figure}[H]
			\centering
			\includegraphics[scale=0.4]{../../Libro ML/Decision Tree/notas/img/arbol3-4.pdf}
			\caption{Separación de la rama izquierda del árbol de la figura \ref{fig:arbol3-2} mediante dolor de pecho.}
			\label{fig:arbol3-4}
		\end{figure}
	\end{itemize}
\end{frame}

\begin{frame}{Ejemplo 3}
	\begin{itemize}
		\item Mediante esta separación hay 111 pacientes que sufren dolor de pecho de los cuales 13 
			sufrieron ataque y 98 no, y hay 53 pacientes que no sufren dolor de pecho de los cuales 
			24 sufrieron un ataque y 29 no. \pause
		\item Los cálculos de la impureza para esta separación son los siguientes: \pause
	\end{itemize}
\begin{align*}
	G_{Si} &= 1 - \left(\dfrac{13}{13+98}\right)^2 - \left(\dfrac{98}{13+98}\right)^2 = 0.2068\\
	G_{No} &= 1 - \left(\dfrac{24}{24+29}\right)^2 - \left(\dfrac{29}{24+29}\right)^2 = 0.4955\\
	G_{DP} &= \dfrac{13+98}{13+98+24+29}\times 0.2068 + \dfrac{24+29}{13+98+24+29}\times 0.4955\\
	G_{DP} &= 0.3001
\end{align*}
\end{frame}

\begin{frame}{Ejemplo 3}
	\begin{itemize}
		\item Ahora se realiza la separación mediante arterias bloqueadas.\pause
		\item Mediante esta separación de los 164 pacientes 49 padecen de arterias bloqueadas de los cuales 24 de ellos sufrieron 
			ataque y 25 no\pause,
		\item Mientras que de los 115 restantes que no padecen de arterias bloqueadas 13 sufrieron ataque y 102 no.\pause
		\item Observe la figura \ref{fig:arbol3-5}.\pause
	\end{itemize}
\end{frame}

\begin{frame}{Ejemplo 3}
	\begin{figure}[H]
		\centering
		\includegraphics[scale=0.55]{../../Libro ML/Decision Tree/notas/img/arbol3-5.pdf}
		\caption{Separación de la rama izquierda del árbol de la figura \ref{fig:arbol3-2} mediante arterias bloqueadas.}
		\label{fig:arbol3-5}
	\end{figure}
\end{frame}

\begin{frame}{Ejemplo 3}
	Los cálculos correspondientes son los siguientes:\pause
	\begin{align}
		\label{eq:impurezaABSi}
		G_{Si} &= 1 - \left(\dfrac{24}{24+25}\right)^2 - \left(\dfrac{25}{24+25}\right)^2 = 0.4997\\
		\label{eq:impurezaABNo}
		G_{No} &= 1 - \left(\dfrac{13}{13+102}\right)^2 - \left(\dfrac{102}{13+102}\right)^2 = 0.2005\\
		G_{AB} &= \dfrac{24+25}{24+25+13+102}\times 0.4997 + \dfrac{13+102}{24+25+13+102} = 0.2899\nonumber
	\end{align}
\end{frame}

\begin{frame}{Ejemplo 3}
	\begin{itemize}
		\item Dado que $G_{AB}$ es menor que $G_{DP}$ se elige arterias bloqueadas para realizar la 
			separación de los pacientes.\pause
		\item De esta forma el árbol resultante hasta ahora se muestra en la figura \ref{fig:arbol3-6}.
	\end{itemize}
\end{frame}

\begin{frame}{Ejemplo 3}
	\begin{figure}[H]
		\centering
		\includegraphics[scale=0.4]{../../Libro ML/Decision Tree/notas/img/arbol3-6.pdf}
		\caption{El árbol de decisión con sólo una rama separada mediante dos criterios.}
		\label{fig:arbol3-6}
	\end{figure}
\end{frame}

\begin{frame}{Ejemplo 3}
	\begin{itemize}
		\item Ahora se debe evaluar el separar los pacientes en la rama izquierda del árbol de la figura \ref{fig:arbol3-6} 
			mediante el dolor de pecho.\pause 
		\item Esto se determina calculando la impureza mediante la separación a través del dolor de pecho\pause
		\item Si esta impureza resulta menor que la obtenida al separar por arterias bloqueadas entonces se realiza\pause
		\item Si no fuera menor entonces no se realiza la separación.\pause 
		\item Recuerde que el objetivo al separar es obtener la menor impureza posible.
	\end{itemize}
\end{frame}

\begin{frame}{Ejemplo 3}
	\begin{itemize}
		\item Considere que de los 49 pacientes que sufrieron de un ataque y padecen de arterias bloqueadas, 
			se sabe que 20 padecen de dolor de pecho y 29 no.\pause
		\item De los 20 que padecen de dolor de pecho 17 sufrieron un ataque y 3 no. \pause
		\item Mientras que de los 29 que no padecen dolor de pecho 7 sufrieron un ataque y 22 no.\pause
		\item La figura \ref{fig:arbol3-7} muestra esta clasificación.
	\end{itemize}
\end{frame}

\begin{frame}{Ejemplo 3}
	\begin{figure}[H]
		\centering
		\includegraphics[scale=0.6]{../../Libro ML/Decision Tree/notas/img/arbol3-7.pdf}
		\caption{Separación de la rama izquierda afirmativa del árbol de la figura \ref{fig:arbol3-6} mediante dolor de pecho.}
		\label{fig:arbol3-7}
	\end{figure}
\end{frame}

\begin{frame}{Ejemplo 3}
	Los cálculos de la impureza de esta separación son los siguientes:\pause
	\begin{align*}
		G_{Si} &= 1 - \left(\dfrac{17}{17+3}\right)^2 - \left(\dfrac{3}{17+3}\right)^2 = 0.255\\
		G_{No} &= 1 - \left(\dfrac{7}{7+22}\right)^2 - \left(\dfrac{22}{7+22}\right)^2 = 0.3662\\
		G_{DP} &= \dfrac{17+3}{17+3+7+22}\times 0.255 + \dfrac{7+22}{17+3+7+22}\times 0.3662 = 0.3208
	\end{align*}
\end{frame}

\begin{frame}{Ejemplo 3}
	\begin{itemize}
		\item En este punto se debe decidir si los últimos nodos son terminales. \pause
		\item Si la impureza $G_{DP}$ de la separación mediante dolor de pecho de la figura \ref{fig:arbol3-7} es menor 
			que la impureza anterior $G$ calculada en \ref{eq:impurezaABSi} entonces se hace la separación, 
			si no es menor entonces estos nodos son terminales. \pause 
		\item Dado que $G_{DP}=0.3208$ es menor que $G_{Si}=0.4997$ de la expresión \ref{eq:impurezaABSi}, se realiza la separación.
	\end{itemize}
\end{frame}

\begin{frame}{Ejemplo 3}
	\begin{itemize}
		\item Ahora se repiten los cálculos y la comparación con la impureza anterior para la rama negativa, 
		es decir los pacientes que sufren de buena circulación pero no de arterias bloqueadas.\pause
		\item De estos 115 pacientes 33 sufren de dolor de pecho y 82 no. \pause 
		\item De los 33 que sufren dolor de pecho 7 tuvieron un ataque y 26 no\pause, 
		\item Por otro lado, de los 82 que no sufren de dolor de pecho 6 tuvieron un ataque y 76 no.\pause
		\item La figura \ref{fig:arbol3-8} muestra estos datos de la separación.
	\end{itemize}
\end{frame}

\begin{frame}{Ejemplo 3}
	\begin{figure}[H]
		\centering
		\includegraphics[scale=0.6]{../../Libro ML/Decision Tree/notas/img/arbol3-8.pdf}
		\caption{Separación de la rama izquierda negativa del árbol de la figura \ref{fig:arbol3-6} mediante dolor de pecho.}
		\label{fig:arbol3-8}
	\end{figure}
\end{frame}

\begin{frame}{Ejemplo 3}
	Los cálculos de la impureza de esta separación son los siguientes:
	\begin{align*}
		G_{Si} &= 1 - \left(\dfrac{7}{7+26}\right)^2 - \left(\dfrac{26}{7+26}\right)^2 = 0.3342\\
		G_{No} &= 1 - \left(\dfrac{6}{6+76}\right)^2 - \left(\dfrac{76}{6+76}\right)^2 = 0.1356\\
		G_{DP} &= \dfrac{7+26}{7+26+6+76}\times 0.3342 + \dfrac{6+76}{7+26+6+76}\times 0.1356 = 0.1926
	\end{align*}
\end{frame}

\begin{frame}{Ejemplo 3}
	\begin{itemize}
		\item Dado que $G_{DP}$ es menor que la impureza anterior calculada en la expresión \ref{eq:impurezaABNo}, 
			también se realiza la separación en esta rama.\pause
		\item Por lo tanto, el árbol con las separaciones realizadas en su rama izquierda se muestra en la figura 
			\ref{fig:arbol3-9}.
	\end{itemize}
\end{frame}

\begin{frame}{Ejemplo 3}
	\begin{figure}[H]
		\centering
		\includegraphics[scale=0.3]{../../Libro ML/Decision Tree/notas/img/arbol3-9.pdf}
		\caption{Separación completa de la rama izquierda del árbol.}
		\label{fig:arbol3-9}
	\end{figure}
\end{frame}

\begin{frame}{Ejemplo 3}
	\begin{itemize}
		\item A continuación solo falta repetir el mismo procedimiento para la rama derecha del árbol de la figura \ref{fig:arbol3-9}, 
			el caso en el que los pacientes no tienen buena circulación. \pause
		\item Se debe calcular la impureza para dolor de pecho y arterias bloqueadas, decidir por la menor impureza y si conviene o no la separación por el criterio
			correspondiente.\pause 
		\item Realizando los cálculos restantes los cálculos, el árbol de decisión resultante se muestra en la figura \ref{fig:arbol3-10}. 
	\end{itemize}
\end{frame}

\begin{frame}{Ejemplo 3}
	\begin{figure}[H]
		\centering
		\includegraphics[scale=0.3]{../../Libro ML/Decision Tree/notas/img/arbol3-10.pdf}
		\caption{Árbol de decisión final para el ejemplo de los pacientes del estudio de la tabla \ref{tab:datosPacientes}.}
		\label{fig:arbol3-10}
	\end{figure}
\end{frame}

\section{Bibliografía}
\begin{frame}[allowframebreaks]{References}
    \nocite{*}
    \bibliographystyle{plain}
    \bibliography{biblioDT}
\end{frame}

\end{document}
