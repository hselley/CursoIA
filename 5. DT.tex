\documentclass[11pt,aspectratio=169]{beamer}
\usetheme{Boadilla}
\usecolortheme{beaver}
\usepackage[utf8]{inputenc}
\usepackage[spanish,mexico]{babel}
\usepackage{amsmath}
\usepackage{amsfonts}
\usepackage{amssymb}
\usepackage{graphicx}
\usepackage{booktabs}
\usepackage{listings}
\usepackage{algorithm,algorithmic}
\author{Héctor Selley}
\title{Árboles de decisión}
\providecommand{\tightlist}{%
  \setlength{\itemsep}{0pt}\setlength{\parskip}{0pt}}
%\setbeamercovered{transparent} 
%\setbeamertemplate{navigation symbols}{} 
%\logo{} 
\institute{Universidad Anáhuac México} 
\date{\today} 
%\subject{} 
\begin{document}

\AtBeginSection[] % Do nothing for \section*
{
\begin{frame}<beamer>
\frametitle{Contenido}
\tableofcontents[currentsection]
\end{frame}
}

\begin{frame}
	\titlepage
\end{frame}

%\begin{frame}
%	\tableofcontents
%\end{frame}

\section{¿Qué es un árbol de decisión?}
\begin{frame}{¿Qué es un árbol de decisión?}
	\begin{itemize}\pause
		\item Un árbol de decisión es un modelo de predicción. \pause
		\item Se utiliza en diversas disciplinas como la Inteligencia Artificial, 
		Medicina, Ingeniería, Ciencia de Datos y la Economía, entre muchas otras. \pause
		\item Los árboles se construyen desde un conjunto de datos \pause 
		\item Los diagramas resultantes son similares a los sistemas de predicción que se basan
		en reglas \pause
		\item Sirven para categorizar una serie de condiciones que ocurren en forma sucesiva.
	\end{itemize}
\end{frame}

\begin{frame}{¿Qué es un árbol de decisión?}
	Los árboles de decisión se utilizan en cualquier proceso que implique una toma de decisión\pause, por ejemplo:\pause
	\begin{itemize}
		\item Búsqueda binaria \pause
		\item Sistemas expertos \pause
		\item Árboles de juego
	\end{itemize}	
\end{frame}

\begin{frame}{¿Qué es un árbol de decisión?}
	\begin{itemize}
		\item Los árboles de decisión son generalmente binarios \pause
		\item Significa que pueden tomar dos opciones \pause
		\item Aunque es posible que existan árboles de tres o más opciones.
	\end{itemize}
\end{frame}

\begin{frame}{¿Para qué sirve un árbol de decisión?}
	Objetivos del árbol de decisión: \pause
	\begin{itemize}
		\item Encontrar un árbol binario que clasifique datos de entrada 
			con una \textit{dispersión} mínima. \pause
		\item Calcular la eficiencia del proceso de clasificación mediante 
			la \textit{dispersión}.
	\end{itemize}	
\end{frame}

\begin{frame}{¿Qué es un árbol de decisión?}
	\begin{block}{Árbol de decisión}
		Árbol de decisión es una técnica de estructura de datos jerárquicos 
		que se utiliza para la clasificación y regresión de datos.\pause 
		Este método emplea la técnica \textit{divide y vencerás}, mediante 
		la cual encuentra recursivamente la separación por clasificación de 
		los datos de entrada. 
	\end{block}
\end{frame}

\begin{frame}{¿Qué es un árbol de decisión?}
	\begin{itemize}
		\item Un árbol de decisión es un grafo que consiste en nodos y aristas.\pause
		\item Cada nodo puede tener máximo dos aristas, razón por lo que se le denomina 
		como binario.\pause
		\item Un árbol de decisión responde una pregunta acerca de los datos y los 
		clasifica de acuerdo con la respuesta de dicha pregunta. \pause
	\end{itemize}

	Utilizaremos algunos ejemplos para explicar los árboles de decisión, cómo se 
	definen y construyen.
\end{frame}

\section{Ejemplos}
\subsection{Ejemplo 1}
\begin{frame}{Ejemplo 1}
	La figura \ref{fig:arbol1} muestra un árbol de decisión que mediante una pregunta, 
	cuya respuesta puede ser verdadero o falso, clasifica los datos de entrada en dos 
	grupos. \pause

	\begin{figure}[H]
		\centering
		\includegraphics[scale=0.5]{../../Libro ML/Decision Tree/notas/img/arbol1.pdf}
		\caption{Ejemplo de árbol de decisión.}
		\label{fig:arbol1}
	\end{figure}
\end{frame}

\begin{frame}{Ejemplo 1}
	\begin{itemize}
		\item En los árboles, los \textbf{nodos} se representan con círculos o elipses 
		en los cuales se aloja una pregunta \pause 
		\item Las aristas son la conexión entre ellos a través de la respuesta de la 
		pregunta. 
		\item Se denomina \textbf{rama} al conjunto de al menos dos nodos conectados 
		por una arista.
	\end{itemize}
\end{frame}

\begin{frame}{Ejemplo 1}
	\begin{itemize}
		\item Imagine que tiene un conjunto de datos que desea clasificar mediante 
		una pregunta cuya respuesta es verdadero o falso. (figura \ref{fig:arbol1}) \pause
		\item Esto permite clasificar los datos en dos grupos, uno cuya respuesta 
		fue verdadera y otro cuya respuesta fue falsa. \pause
		\item Para un árbol tan pequeño como el de este ejemplo, la separación de 
		los datos es muy limitada por lo que se busca mejorarla empleando más nodos 
		en el árbol, lo que significa un mayor número de categorías.\pause
		\item Adicionalmente, la pregunta sólo acepta respuestas absolutas, si se 
		requiere de un rango de respuestas, por ejemplo, un rango de números 
		habría que modificar el árbol.
	\end{itemize}
\end{frame}

\subsection{Ejemplo 2}
\begin{frame}{Ejemplo 2}
	Construyamos un árbol con más nodos y ramificaciones, para este ejemplo se clasifica una persona de acuerdo con su edad. Se clasifica a una persona como adulto si su 
edad es mayor o igual a 18 años, como adolescente si está entre los 12 y 18 años, como niño si está entre los 2 y 12 años y como bebé si es menor a 2 años. El árbol 
resultante se muestra en la figura \ref{fig:arbol2}.

En el árbol resultante de la figura \ref{fig:arbol2} clasifica a las personas de acuerdo con su edad utilizando el criterio antes mencionado. En el árbol las personas 
han sido clasificadas en los nodos: adulto, adolescente, niño y bebé. A estos nodos, aquellos que tienen flechas que llegan a él pero no salen de él, se les denomina 
como \textbf{nodos terminales} o de decisión. Al nodo inicial, aquel del cual sólo salen flechas de el pero no entran, se le denomina \textbf{nodo raíz} o simplemente
\textbf{raíz}. Los demás simplemente se les denomina como \textbf{nodos}. 

\begin{figure}[H]
	\centering
	\includegraphics[scale=0.5]{../../Libro ML/Decision Tree/notas/img/arbol2.pdf}
	\caption{Árbol de decisión con más nodos.}
	\label{fig:arbol2}
\end{figure}
\end{frame}

\begin{frame}{Ejemplo 2}
	En la figura \ref{fig:arbol2} el nodo en rojo es la raíz, los nodos en verde son terminales y los azules son simplemente nodos.

En el árbol de decisión de la figura \ref{fig:arbol2} clasifica adecuadamente a las personas, dado que una persona sólo tiene una edad, la clasificación es 
perfecta de esa forma. Imagínese que deseamos clasificar personas de acuerdo con otro criterio, un criterio en el cual la respuesta no será tan específica como la 
edad o incluso puede que no haya una respuesta. Por ejemplo, imagine que deseamos clasificar personas de acuerdo con su sabor preferido de helado, puede que tenga 
uno, varios o incluso ninguno.
En una situación como ésta, habrá una \textbf{impureza} en la clasificación. Más adelante se explicará a través del ejemplo la impureza y cómo se calcula.
\end{frame}

\subsection{Ejemplo 3}
\begin{frame}{Ejemplo 3}
	
\end{frame}

\begin{frame}{}
	
\end{frame}

\begin{frame}{}
	
\end{frame}

\begin{frame}{}
	
\end{frame}

\begin{frame}{}
	
\end{frame}

\begin{frame}{}
	
\end{frame}

\begin{frame}{}
	
\end{frame}

\begin{frame}{}
	
\end{frame}

\begin{frame}{}
	
\end{frame}

\begin{frame}{}
	
\end{frame}

\begin{frame}{}
	
\end{frame}







\begin{frame}{}
	
\end{frame}

\end{document}
