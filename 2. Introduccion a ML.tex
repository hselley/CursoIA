\documentclass[11pt,aspectratio=169]{beamer}
\usetheme{Boadilla}
\usecolortheme{beaver}
\usepackage[utf8]{inputenc}
\usepackage[spanish,mexico]{babel}
\usepackage{amsmath}
\usepackage{amsfonts}
\usepackage{amssymb}
\usepackage{graphicx}
\author{Héctor Selley}
\title{Introducción al aprendizaje automático}
\providecommand{\tightlist}{%
  \setlength{\itemsep}{0pt}\setlength{\parskip}{0pt}}
%\setbeamercovered{transparent} 
%\setbeamertemplate{navigation symbols}{} 
%\logo{} 
\institute{Universidad Anáhuac México} 
\date{\today}
%\subject{} 
\begin{document}

\AtBeginSection[] % Do nothing for \section*
{
\begin{frame}<beamer>
\frametitle{Outline}
\tableofcontents[currentsection]
\end{frame}
}

\begin{frame}
	\titlepage
\end{frame}

\begin{frame}
	\tableofcontents
\end{frame}

\section{Preámbulo}
\begin{frame}{Inteligencia artificial}\pause
\begin{block}{}
Combinación de algoritmos planteados con el propósito de crear máquinas que presenten las mismas capacidades que el ser humano.
\end{block}\pause
\begin{itemize}
	\item En informática, la inteligencia expresada por máquinas, sus procesadores y su software, que serían los análogos al cuerpo, el cerebro y 
		la mente, respectivamente, a diferencia de la inteligencia natural demostrada por humanos y ciertos animales con cerebros complejos.\pause
	\item En ciencias de la computación, una máquina \textit{inteligente} ideal es un agente flexible que percibe su entorno y lleva a cabo acciones que 
		maximicen sus posibilidades de éxito en algún objetivo o tarea.\pause
	\item En 1956, John McCarthy acuñó la expresión \textit{inteligencia artificial}, y la definió como \textit{la ciencia e ingenio de hacer máquinas 
		inteligentes, especialmente programas de cómputo inteligentes}.
\end{itemize}
\end{frame}

\begin{frame}{Machine Learning}
\begin{itemize}
	\item Creación de sistemas computarizados y algoritmos que permiten a las máquinas ``aprender'' de situaciones previas. \pause
	\item Es una forma de la IA que permite a un sistema aprender de los datos en lugar de aprender mediante la programación explícita. \pause
	\item Machine learning no es un proceso sencillo.\pause
	\item Conforme el algoritmo ingiere datos de entrenamiento, es posible producir modelos más precisos basados en datos. \pause
	\item Un modelo de machine learning es la salida de información que se genera cuando entrena su algoritmo de machine learning con datos. \pause
	\item Después del entrenamiento, al proporcionar un modelo con una entrada, se le dará una salida.\pause
	\item Por ejemplo, un algoritmo predictivo creará un modelo predictivo. 
\end{itemize}
\end{frame}

\begin{frame}{Machine Learning$_2$}
\begin{itemize}
	\item Permite modelos a entrenar con conjuntos de datos antes de ser implementados. \pause
	\item Algunos modelos de machine learning están online y son continuos. \pause
	\item Después de que un modelo ha sido entrenado, se puede utilizar en tiempo real para aprender de los datos. \pause
	\item Las mejoras en la precisión son el resultado del proceso de entrenamiento y la automatización que forman parte del machine learning.
\end{itemize}
\end{frame}

\begin{frame}{Machine Learning$_3$}
\begin{itemize}
	\item ML consiste en extraer conocimiento de los datos. \pause
	\item Es un área de estudio en la intersección de la estadística, inteligencia artificial y la ciencia de la computación.\pause
	\item También se le conoce como análisis predictivo o aprendizaje estadístico. \pause
	\item Los métodos de ML se han convertido en parte de la vida diaria, van desde recomendaciones automatizadas de películas, 
		comida, productos, estaciones de radio en linea, vídeos e incluso el reconocimiento de personas en fotografías en redes sociales\pause
	\item Al utilizar servicios como Amazon, Facebook, Netflix o Google (entre otros) se está utilizando también algún modelo de ML.
\end{itemize}
\end{frame}

\begin{frame}{Machine Learning$_4$}
\begin{figure}[H]
	\centering
	\includegraphics[scale=0.45]{../Programacion para ML/Contenido/img/AI-ML.png}
\end{figure}
\end{frame}

\begin{frame}{¿Porque usar Machine Learning?}
\begin{itemize}
	\item Es necesario utilizar Machine Learning para que la toma de decisiones sea más rápida.\pause
	\item El paradigma ML utiliza datos y resultados esperados (si los hay) y emplea la computadora para construir un programa, que se 
		conoce como \textbf{modelo}.\pause
	\item Este modelo puede ser utilizado para la toma de decisiones y obtener resultados para nuevas entradas.
\end{itemize}
\end{frame}

\begin{frame}{Modelo de ML}
Para construir un modelo de ML, se deben seguir los siguientes pasos:\pause
\begin{itemize}
	\item Obtener datos o registros (suficientes) para conocer la historia en ellos y almacenarlos en algún medio apropiado 
		(bases de datos, registros, archivos de texto plano, CSV, etc). \pause
	\item Elegir atributos clave en los datos que puedan ser útiles para la construcción del modelo. \pause
	\item Observar y capturar los atributos y su comportamiento durante varios períodos de tiempo, lo que incluye el comportamiento normal y anómalo. \pause
		Estos resultados serían las salidas y los datos las entradas.\pause
	\item Alimentar algún algoritmo de ML con este conjunto de entradas y salidas, construir un modelo que aprenda los patrones inherentes 
		y observar el producto o resultado correspondiente.\pause
	\item Implementar este modelo de modo que, para nuevos datos, pueda predecir si el comportamiento es normal o anómalo.
\end{itemize}
\end{frame}

\begin{frame}{Definición Formal de Machine Learning}
\begin{itemize}
	\item Para definir Machine Learning es necesario volver al inicio, a las bases definidas por el renombrado profesor Tom Mitchell en 
		1997\footnote{Científico computacional estadounidense y profesor en la Universidad E. Fredkin en la Universidad Carnegie Mellon (CMU) 
		\url{https://en.wikipedia.org/wiki/Tom_M._Mitchell}}.\pause
	\item  La idea de Machine Learning es que un algoritmo de aprendizaje ayudará a aprender a partir de los datos. \pause
	\item El profesor Mitchell lo define de la siguiente forma:\pause
\end{itemize}
\begin{block}{}
\textit{"
Se dice que un programa computacional aprende desde la experiencia \textbf{E} con respecto a una clase de tareas \textbf{T} y una medida de desempeño \textbf{P}, 
si su desempeño en llevar a cabo las tareas \textbf{T}, medido por \textbf{P}, mejora con la experiencia \textbf{E}.
"}
\end{block}
\end{frame}

\begin{frame}{Definición Formal de Machine Learning}
Los parámetros T, P y E son los componentes principales de cualquier algoritmo de aprendizaje:\pause
\begin{figure}[H]
	\centering
	\includegraphics[scale=0.25]{../Programacion para ML/Contenido/img/TPE.png}
\end{figure}
\end{frame}

\begin{frame}{Definición Formal de Machine Learning}
Se puede simplificar la definición de la siguiente manera, Machine Learning es la disciplina que consiste en algoritmos de aprendizaje que:\pause
\begin{itemize}
	\item Mejoren el desempeño \textit{P}\pause
	\item Al ejecutar una tarea \textit{T}\pause
	\item A través del tiempo con la experiencia \textit{E}.
\end{itemize}
\end{frame}

\begin{frame}{Definición de la tarea \textit{T}}
\begin{itemize}
	\item La tarea \textit{T} es básicamente el problema del mundo real a resolver, puede ser desde encontrar la mejor estrategia de marketing 
		hasta predecir fallas estructurales.\pause
	\item Lo más recomendable es definir la tarea de la forma más concreta posible.\pause
 	\item Las tareas basadas en ML son difíciles de resolver mediante la programación convencional y el enfoque tradicional.\pause
	\item Una tarea \textit{T} generalmente se puede definir como una tarea de aprendizaje automático en función del proceso o flujo de trabajo 
		que el sistema debe seguir para operar con datos puntuales o muestras. 
\end{itemize}
\end{frame}

\begin{frame}{Definición de la tarea \textit{T}}
La siguiente lista muestra algunas tareas comunes:
\begin{description}
	\item[Clasificación o categorización.] Abarca la lista de tareas o problemas donde la maquina toma como entrada datos puntuales o muestras y asigna
		una clase específica o categoría a cada muestra. \pause Un ejemplo sería clasificar animales a partir de imágenes.\pause
	\item[Regresión.] Involucra realizar una predicción tal que para una entrada le corresponde como salida un valor numérico en lugar 
		de una clase o categoría.\pause Por ejemplo, predecir los precios de la vivienda considerando el área del terreno, el número de pisos, baños y recámaras
		como atributos de entrada para cada dato puntual.\pause
	\item[Detección de anomalías. ] Implica analizar registros de eventos o transacciones y otros datos puntuales para encontrar anomalías,
		patrones inusuales o eventos distintos a un comportamiento normal. \pause Por ejemplo, detección de ataques de denegación de servicio (DoS) o fraudes.
\end{description}
\end{frame}

\begin{frame}{Definición de la tarea \textit{T}}
\begin{description}
	\item[Anotación estructurada. ] Implica realizar algún análisis en los datos puntuales de entrada y añadir anotaciones como metadatos estructurados. \pause
		a los datos originales, donde las anotaciones representan información adicional e incluso relaciones entre los datos. \pause Por ejemplo, anotaciones de texto 
		a algunas partes de un discurso, dichas anotaciones podrían ser correcciones gramaticales o sentimientos. \pause		
	\item[Traducción. ] Las tareas de traducción automática suelen ser del tipo en el que si se tienen muestras de datos de entrada pertenecientes a un idioma específico, 
		realizará la traducción del texto a ese idioma.\pause La traducción basada en lenguaje natural es definitivamente un área enorme que maneja una gran cantidad de datos
		de texto.
\end{description}
\end{frame}

\begin{frame}{Definición de la tarea \textit{T}}
\begin{description}
	\item[Agrupamiento o agrupación. ] Los clústeres o grupos se forman a partir de muestras de datos de entrada y haciendo que la máquina aprenda u observe 
		patrones, relaciones o similitudes inherentes en los datos de entrada. Por lo general, no se tienen datos preetiquetados para estas tareas. \pause 
		Por ejemplo agrupar productos, eventos y entidades similares.\pause
	\item[Transcripciones. ] Estas tareas involucran tomar varias representaciones de datos, que son usualmente continuas y no estructuradas, y convertirlas en datos
		discretos y estructurados.\pause Por ejemplo traducción de datos de voz a texto o reconocimiento de texto, entre otros.
\end{description}
\end{frame}

\begin{frame}{Definición de la experiencia \textit{E}}
\begin{itemize}
	\item El proceso de consumir un conjunto de datos (muestras o datos puntuales) de manera que un algoritmo de aprendizaje o modelo aprenda patrones inherentes
		se conoce como experiencia \textit{E}.\pause
	\item Es posible alimentar el algoritmo con muestras de datos de una sola vez utilizando datos históricos o incluso proporcionar muestras de datos nuevos
		cada vez que se adquieren.\pause
	\item un modelo o algoritmo obtiene experiencia generalmente ocurre como un proceso iterativo, también conocido como \textbf{entrenamiento}. \pause
	\item Se podría pensar en el modelo como una entidad tal como un ser humano que adquiere conocimiento o experiencia a través de la observación de datos y 
		aprendiendo sobre varios atributos, relaciones y patrones presentes en los datos. 
\end{itemize}
\end{frame}

\begin{frame}{Definición del desempeño \textit{P}}
\begin{itemize}
	\item Supongamos que tenemos un algoritmo de Machine Learning que realiza una tarea \textit{T} y obtiene experiencia \textit{E} con datos puntuales 
		durante un periodo de tiempo.\pause
	\item Ahora se debe determinar si su desempeño es bueno o bien si está comportando correctamente, esto lo determinamos con el desempeño \textit{P}.\pause
	\item El desempeño \textit{P} es una métrica o medida cuantitativa que se utiliza para determinar que tan bien funciona el algoritmo o modelo que está 
		realizando la tarea \textit{T} con experiencia \textit{E}.\pause
	\item Las medidas típicas de desempeño incluyen exactitud, precisión, recuperación, puntaje F1, sensibilidad, especificidad, tasa de error, tasa de 
		clasificación errónea entre otras.
\end{itemize}
\end{frame}

\begin{frame}{Definición del desempeño \textit{P}}
\begin{itemize}
	\item Las medidas de rendimiento generalmente se evalúan en un conjunto de datos de entrenamiento, así como otro conjunto no visto o aprendido antes, 
		a esto se le conoce como muestras de datos de prueba y validación.\pause
	\item La idea detrás de esto es generalizar el algoritmo para que haya un sesgo en el conjunto de entrenamiento y funcione bien en el futuro para otro 
		conjunto de datos diferente.
\end{itemize}
\end{frame}
  
\begin{frame}{Campo multidisciplinario}
Machine Learning es una sub-rama de la Inteligencia Artificial, sin embargo utiliza conceptos que se han derivado o utilizado desde otras áreas del conocimiento. 
La figura  muestra las diversas áreas de conocimiento en las que se apoya el Machine Learning.\pause
\begin{figure}[H]
	\centering
	\includegraphics[scale=0.3]{../Programacion para ML/Contenido/img/ML-multi.png}
\end{figure}
\end{frame}	

\begin{frame}{Inteligencia Artificial y Machine Learning}\pause

\begin{itemize}
	\item Machine Learning es una disciplina científica del campo de la Inteligencia Artificial\pause
	\item Machine Learning crea sistemas que aprenden automáticamente en el tiempo \pause
	\item ¿Qué es aprender en ML?\pause
	\begin{itemize}
		\item Usar algoritmos para construir modelos que revelan patrones en los datos a partir de su historia\pause
		\item Se basa principalmente en el aprendizaje estadístico
	\end{itemize}
\end{itemize}
\end{frame}

\begin{frame}{Inteligencia Artificial y Machine Learning$_2$}
\begin{figure}[H]
	\centering
	\includegraphics[scale=0.375]{../Programacion para ML/Contenido/img/AI-ML2.png}
\end{figure}
\end{frame}

\begin{frame}{Inteligencia Artificial y Machine Learning$_3$}
\begin{figure}[H]
	\centering
	\includegraphics[scale=0.375]{../Programacion para ML/Contenido/img/AI-ML3.png}
\end{figure}
\end{frame}

\begin{frame}{Métodos de Machine Learning}
ML tiene múltiples algoritmos, técnicas y metodologías que se pueden utilizar para construir modelos para resolver problemas del mundo
real utilizando datos.\pause Las siguientes son algunas de las principales áreas de los métodos de Machine Learning.\pause

\begin{itemize}
	\item Métodos basados en la cantidad de supervisión humana en el proceso de aprendizaje\pause
		\begin{itemize}
			\item Aprendizaje supervisado\pause
			\item Aprendizaje no supervisado\pause
			\item Aprendizaje semi-supervisado\pause
			\item Aprendizaje por refuerzo
		\end{itemize}
\end{itemize}
\end{frame}

\begin{frame}{Métodos de Machine Learning}
\begin{itemize}

	\item Métodos basados en la habilidad de aprender desde un conjunto de datos incrementales\pause
		\begin{itemize}
			\item Aprendizaje por lotes\pause
			\item Aprendizaje en línea\pause
		\end{itemize}
	\item Métodos basados en su enfoque de generalización a partir de un conjunto de datos\pause
		\begin{itemize}
			\item Aprendizaje basado en instancias\pause
			\item Aprendizaje basado en modelos
		\end{itemize}
\end{itemize}
\end{frame}

\section{Aprendizaje supervisado}
\begin{frame}{Aprendizaje supervisado}
\begin{itemize}
	\item Los algoritmos o métodos de aprendizaje supervisado utilizan un conjunto de datos de entrada (conjunto de entrenamiento) y salidas 
		asociadas a ellos (etiquetas) para cada una de las entradas durante el proceso de entrenamiento del modelo.\pause
	\item Utilizan un conjunto de datos de entrada (conjunto de entrenamiento) y salidas asociadas a ellos (etiquetas) para cada una de las 
		entradas durante el proceso de entrenamiento del modelo. \pause
	\item El principal objetivo es aprender a asociar entre un conjunto de entradas $\textbf{x}$ y su salida correspondiente $\textbf{y}$ 
		basado en el entrenamiento mediante múltiples instancias.\pause
	\item El conocimiento adquirido puede ser utilizado en el futuro para predecir una salida $\textbf{y'}$ para nuevos datos de entrada $\textbf{x'}$. \pause
	\item A este proceso se le llama \textbf{supervisado} dado que el modelo aprende de muestras en las que se conoce previamente su salida correspondiente.
\end{itemize}
\end{frame}

\begin{frame}{Aprendizaje supervisado}
\begin{itemize}
	\item El aprendizaje supervisado comienza típicamente con un conjunto establecido de datos y una cierta comprensión de cómo se clasifican estos datos. \pause
	\item El aprendizaje supervisado tiene la intención de encontrar patrones en datos que se pueden aplicar a un proceso de analítica.\pause
	\item Estos datos tienen características etiquetadas que definen el significado de los datos.\pause
	\item Por ejemplo, se puede crear una aplicación de machine learning con base en imágenes y descripciones escritas que distinga entre millones de animales.
\end{itemize}

\begin{figure}[H]
	\centering
	\includegraphics[scale=0.25]{../Programacion para ML/Contenido/img/supervisado.png}
\end{figure}
\end{frame}

\begin{frame}{Aprendizaje supervisado}
\begin{itemize}
	\item Aprender de datos que cuentan con una clasificación (etiquetas)\pause
	\item Problemas de clasificación: reconocimiento en imágenes, diagnóstico médico, detección de fraudes y correo spam, procesamiento de 
		lenguaje natural (NLP), biometría\pause
	\item Problemas de regresión: optimización de procesos, predicciones financiaras, forecasting, crecimiento estimado de la población, 
		impacto de mercadotecnia\pause
\end{itemize}

\begin{figure}[H]
	\centering
	\includegraphics[scale=0.2]{../Programacion para ML/Contenido/img/ejemploSupervisado.png}
\end{figure}
\end{frame}

\begin{frame}{Tipos de aprendizaje supervisado}
En los métodos de aprendizaje supervisado existen dos tipos principales de clases de tareas: \pause
\begin{itemize}
	\item Clasificación.\pause
	\item Regresión.
\end{itemize}
\end{frame}

\begin{frame}{Clasificación}
\begin{itemize}
	\item El principal objetivo es predecir etiquetas de salida que sean categóricas en su naturaleza para un conjunto de datos de entrada,
		empleando un modelo que haya aprendido en la fase de entrenamiento.\pause
	\item Las etiquetas de salida (clases) son naturalmente categóricas lo que significa que son valores discretos no ordenados. \pause
	\item Por ejemplo, si se trata de predecir si el clima será soleado o lluvioso considerando como datos de entrada los atributos: humedad, temperatura, presión
		atmosférica y precipitación.\pause
	\item En este ejemplo sólo hay dos clases, esto es, un problema binario: soleado o lluvioso.
\end{itemize}		
\end{frame}

\begin{frame}{Clasificación}
\begin{figure}[H]
	\centering
	\includegraphics[scale=0.3]{../Programacion para ML/Contenido/img/ML-Supervisado-clasificacion.png}
\end{figure}
\end{frame}

\begin{frame}{Regresión}
\begin{itemize}
	\item Las tareas de ML en las cuales el principal objetivo es la estimación de un valor se les conoce como \textbf{regresiones}. \pause
	\item Los métodos son entrenados con muestras de datos que tienen como salida valores numéricos continuos.\pause
	\item Utilizan atributos de los datos de entrada (también llamados variables explicativas o independientes) y sus valores de 
		salida numéricos continuos (también llamados variable de respuesta, dependiente o de resultado) para aprender relaciones 
		y asociaciones entre las entradas y salidas. \pause 
	\item Con este conocimiento, puede predecir respuestas de salida para instancias de datos nuevos.
\end{itemize}
\end{frame}

\begin{frame}{Regresión}
Un ejemplo sencillo de regresión es la predicción del precio de una casa. Se puede construir un modelo de regresión para predecir los precios utilizando 
datos relativos al área del terreno donde se encuentra la casa. 
\begin{figure}[H]
	\centering
	\includegraphics[scale=0.25]{../Programacion para ML/Contenido/img/ML-Supervisado-regresion.png}
\end{figure}
\end{frame}

\begin{frame}{Regresión}
Algunos de los métodos de regresión son los siguientes:\pause
\begin{itemize}
	\item Regresión lineal simple\pause
	\item Regresión múltiple o multivariable\pause
	\item Regresión polinomial\pause
	\item Regresión no lineal\pause
	\item Regresión lazo (lasso)\pause
	\item Regresión cresta (ridge)\pause
	\item Modelos lineales generalizados			
\end{itemize}
\end{frame}


\section{Aprendizaje no supervisado}
\begin{frame}{Aprendizaje no supervisado}
\begin{itemize}
	\item Estos métodos se llaman no supervisados debido a que el modelo o algoritmo trata por sí solo de aprender estructuras, 
		patrones o relaciones inherentes en los datos, sin ayuda alguna. \pause
	\item Estos métodos son necesarios si no se cuentan con datos previamente etiquetados para el proceso de entrenamiento.\pause
	\item El no supervisado se preocupa más por tratar de extraer conocimientos o información significativa de los datos que por intentar 
		predecir algún resultado.\pause 
	\item Hay más incertidumbre en los resultados del aprendizaje no supervisado, pero también puede obtener mucha información de estos 
		modelos que no es fácil de determinar con solo mirar los datos sin procesar.
\end{itemize}
\end{frame}

\begin{frame}{Tipos de Aprendizaje no supervisado}
Los métodos de aprendizaje no supervisado se pueden clasificar en las siguientes áreas: \pause
\begin{itemize}
	\item Clustering.\pause
	\item Reducción dimensional.\pause
	\item Detección de anomalías.\pause
	\item Reglas de asociación.
\end{itemize}
\end{frame}

\begin{frame}{Clustering}
\begin{itemize}
	\item Son métodos de ML que tratan de encontrar patrones, similitud y relaciones entre los datos y separar estas muestras en diversos grupos.\pause
	\item Estos métodos no están supervisados en absoluto porque intentan agrupar datos observando las características de los datos sin ningún 
		tipo de capacitación, supervisión o conocimiento previo.\pause
	\item Considere un problema del mundo real:
		\begin{itemize}
			\item Se tienen varios servidores en un centro de datos e intentar analizar los registros en busca de problemas o errores típicos.\pause
			\item La tarea principal es determinar los distintos tipos de mensajes de registro que ocurren con frecuencia cada semana.\pause
			\item Se desea agrupar los registros en varios grupos en función de características inherentes. \pause
			\item Un enfoque simple sería extraer características de los registros y agruparlos en función de la similitud en el contenido. 
		\end{itemize}		  
\end{itemize}
\end{frame}

\begin{frame}{Clustering}
\begin{figure}[H]
	\centering
	\includegraphics[scale=0.35]{../Programacion para ML/Contenido/img/ML-NoSupervisado-clustering.png}
\end{figure}
\end{frame}

\begin{frame}{Clustering}
Existen varios tipos de métodos de clustering que pueden ser clasificados de la siguiente forma:\pause
\begin{itemize}
	\item Métodos basados en centroides: K-media o K-mediana.\pause
	\item Métodos de agrupamiento jerárquico como aglomerativo y divisivo (Ward o afinidad de propagación)\pause
	\item Métodos de clustering basados en distribución (modelos de mezclas Gaussianas)\pause
	\item Métodos basados en densidad
\end{itemize}
\end{frame}

\begin{frame}{Reducción dimensional}
\begin{itemize}
	\item Una vez que se comienza a extraer atributos o características de los datos, algunas veces el espacio de
		características se satura con una gran cantidad de atributos o características. \pause
	\item Esto plantea múltiples desafíos, incluido el análisis y la visualización de datos con miles o millones de atributos, 
		lo que hace que el espacio de características sea extremadamente complejo y presente problemas con respecto a los modelos de 
		entrenamiento, la memoria y las limitaciones de espacio.\pause 
	\item Esto se le llama la "\textbf{maldición de la dimensión}".\pause
	\item Los métodos no supervisados se utilizan para reducir el número de atributos o características en los datos. \pause
	\item Existen diversos algoritmos para la reducción dimensional, siendo los más comunes:\pause
		\begin{itemize}
			\item Análisis de Componentes Principales (PCA)\pause
			\item Vecinos Cercanos.\pause
			\item Análisis Lineal Discriminante (LDA).
		\end{itemize}
\end{itemize}
\end{frame}

\begin{frame}{Reducción dimensional}
\begin{figure}[H]
	\centering
	\includegraphics[scale=0.35]{../Programacion para ML/Contenido/img/ML-NoSupervisado-reduccionDimensional.png}
\end{figure}
\end{frame}

\begin{frame}{Detección de anomalías}
\begin{itemize}
	\item El proceso de detección de anomalías, también conocido como detección de \textit{outliers}, consiste en 
		encontrar ocurrencias de eventos inusuales con respecto a un conjunto de datos. \pause
	\item Se puede entrenar el algoritmo con un conjunto de datos que tiene muestras de datos normales sin anomalías.\pause 
	\item Una vez completado el entrenamiento, el algoritmo podrá identificar un dato nuevo como anómalo o normal utilizando 
		su conocimiento aprendido.\pause
\end{itemize}
\end{frame}

\begin{frame}{Detección de anomalías}
\begin{figure}[H]
	\centering
	\includegraphics[scale=0.3]{../Programacion para ML/Contenido/img/ML-NoSupervisado-deteccionAnomalias.png}
\end{figure}
\end{frame}

\begin{frame}{Reglas de asociación}
\begin{itemize}
	\item Son un método de minería de datos que se usa para examinar y analizar grandes conjuntos de datos transaccionales para 
		encontrar patrones y reglas de interés. \pause
	\item Estos patrones representan relaciones y asociaciones interesantes, entre varios artículos a través de transacciones.\pause
	\item Las reglas de asociación también se denomina a menudo análisis de cesta de la compra, que se utiliza para analizar los patrones 
		de compra de los clientes.\pause
	\item Las reglas de asociación ayudan a detectar y predecir patrones transaccionales en función del conocimiento que obtiene de las 
		transacciones de entrenamiento.\pause
	\item Con esta técnica, podemos responder preguntas como qué artículos tienden a comprar juntos las personas.\pause
	\item También podemos asociar o correlacionar productos, por ejemplo, las personas que compran cerveza en un bar también tienden a 
		comprar alitas de pollo.
\end{itemize}
\end{frame}

\begin{frame}{Reglas de asociación}
\begin{figure}[H]
	\centering
	\includegraphics[scale=0.3]{../Programacion para ML/Contenido/img/ML-NoSupervisado-reglaAsociacion.png}
\end{figure}	
\end{frame}

\section{Aprendizaje Semi-Supervisado}
\begin{frame}{Aprendizaje Semi-Supervisado}
\begin{itemize}
	\item Se encuentran entre los métodos de aprendizaje supervisados y no supervisados.\pause
	\item Suelen utilizar una gran cantidad de datos de entrenamiento que no están etiquetados (que forman el componente de aprendizaje no supervisado) 
		y una pequeña cantidad de datos preetiquetados y anotados (que forman el componente de aprendizaje supervisado).\pause
	\item Hay múltiples técnicas disponibles en forma de métodos generativos, métodos basados en grafos y métodos heurísticos.
\end{itemize}
\end{frame}

\begin{frame}{Aprendizaje Semi-Supervisado}
\begin{itemize}
	\item Un enfoque simple sería construir un modelo supervisado basado en datos etiquetados, el cual es limitado, y luego aplicar lo mismo a grandes 
		cantidades de datos no etiquetados para obtener más muestras etiquetadas, entrenar el modelo en ellas y repetir el proceso. \pause
	\item Otro enfoque sería usar algoritmos no supervisados para agrupar muestras de datos similares, usar intervención humana para etiquetar estos grupos 
		y luego usar una combinación de esta información en el futuro. Este enfoque se utiliza en sistemas de etiquetado de imágenes.
\end{itemize}
\end{frame}


\section{Aprendizaje por refuerzo}
\begin{frame}{Aprendizaje por refuerzo}
\begin{itemize}
	\item Los métodos de aprendizaje por refuerzo son un poco diferentes de los métodos convencionales supervisados o no supervisados.\pause
	\item Tenemos un agente que queremos entrenar durante un período de tiempo para interactuar con un entorno específico y mejorar su rendimiento 
		durante un período de tiempo con respecto al tipo de acciones que realiza sobre el entorno.\pause 
	\item El agente comienza con un conjunto de estrategias o políticas para interactuar con el entorno.
\end{itemize}
\end{frame}

\begin{frame}{Aprendizaje por refuerzo}
\begin{itemize}
	\item El aprendizaje de refuerzo es un modelo de aprendizaje conductual.\pause
	\item El algoritmo recibe retroalimentación del análisis de datos, conduciendo el usuario hacia el mejor resultado.\pause
	\item El aprendizaje de refuerzo difiere de otros tipos de aprendizaje supervisado, porque el sistema no está entrenado con el 
		conjunto de datos de ejemplo. Más bien, el sistema aprende a través de la prueba y el error.\pause
	\item Por lo tanto, una secuencia de decisiones exitosas conduce al fortalecimiento del proceso, porque es el que resuelve el 
		problema de manera más efectiva.
\end{itemize}
  \begin{figure}[H]
	\centering
	\includegraphics[scale=0.4]{../Programacion para ML/Contenido/img/refuerzo.png}
	\caption{Aprendizaje por refuerzo}
\end{figure}
\end{frame}


\begin{frame}{Aprendizaje por refuerzo}
\begin{itemize}
	\item Aprender a través de acciones, por lo regular de prueba y error, con el objetivo de maximizar una recompensa o minimizar una penalización. \pause
	\item Recibe retroalimentación del ambiente\pause
	\item Más cercano a metodologías de IA que a algoritmos de ML\pause
	\item Videojuegos, robótica, teoría de juegos, agentes virtuales
\end{itemize}
\end{frame}

\begin{frame}{Aprendizaje por refuerzo}
Los pasos principales de un método de aprendizaje por refuerzo se mencionan a continuación:\pause
\begin{enumerate}
	\item Definir un agente con un conjunto inicial de políticas y estrategias.\pause
	\item Observar el ambiente y el estado actual.\pause
	\item Seleccionar la política óptima y realizar una acción correspondiente.\pause
	\item Obtener la recompensa o penalización correspondiente.\pause
	\item Actualizar las políticas o estrategias si es necesario.\pause
	\item Repetir los pasos 2-5 hasta que el agente aprenda las políticas o estrategias óptimas.
\end{enumerate}
\end{frame}

\begin{frame}{Aprendizaje por refuerzo}
\begin{itemize}
	\item Un ejemplo de aprendizaje por refuerzo consiste en crear un programa que aprenda a jugar ajedrez.\pause
	\item En ese ejemplo el agente sería el programa, el ambiente el tablero, y el estado sería las posiciones de las piezas en el tablero. \pause
\end{itemize}
\begin{figure}[H]
	\centering
	\includegraphics[scale=0.35]{../Programacion para ML/Contenido/img/ML-aprendizajeRefuerzo.png}
\end{figure}	
\end{frame}


\section{Deep Learning}
\begin{frame}{Deep Learning}
\begin{itemize}
	\item Es un método específico de machine learning que incorpora las redes neuronales en capas sucesivas para aprender de los datos de
		manera iterativa.\pause
	\item El deep learning es especialmente útil cuando se trata de aprender patrones de datos no estructurados.\pause
	\item Las redes neuronales complejas de deep learning están diseñadas para emular cómo funciona el cerebro humano, así que las computadoras 
		pueden ser entrenadas para lidiar con abstracciones y problemas mal definidos.\pause
	\item Las redes neuronales y el deep learning se utilizan a menudo en el reconocimiento de imágenes, voz y aplicaciones de visión de computadora.
\end{itemize}
\end{frame}

\begin{frame}{Deep Learning}
\begin{itemize}
	\item Las redes neuronales profundas tienen un enorme potencial para aprender funciones, patrones y representaciones no lineales complejas.\pause
	\item Su poder está impulsando la investigación en múltiples campos, incluida la visión por computadora, el análisis audiovisual, los chatbots 
		y la comprensión del lenguaje natural, por nombrar algunos.
\end{itemize}
\end{frame}

\begin{frame}{Deep Learning}
\begin{figure}[H]
	\centering
	\includegraphics[scale=0.35]{../Programacion para ML/Contenido/img/AI-ML-DL.png}
\end{figure}
\end{frame}

\section{Tipos de problemas que se resuelven con Machine Learning}
\begin{frame}{Aplicaciones representativas de ML}
\begin{figure}[H]
	\centering
	\includegraphics[scale=0.22]{../Programacion para ML/Contenido/img/EjemplosML.png}
\end{figure}
\end{frame}

\section{Diseño de un sistema de aprendizaje}
\begin{frame}{Diseño de un sistema de aprendizaje}
Considerese una estructura en la que se utilice la Big Data, Machine Learning y Cómputo en la nube para la solución de un problema.\pause

\begin{itemize}
	\item \textbf{Big Data}. Permite gestionar grandes volúmenes de información y procesarlos de forma ágil, incluso en tiempo real\pause
	\item \textbf{Cómputo en la nube}. Brinda una infraestructura escalable \pause
	\item \textbf{Machine Learning}. Aprovecha el volumen de información para detectar patrones y comportamientos más acertados
\end{itemize}
\end{frame}

\begin{frame}{Big Data + ML + Cómputo en la nube}
\begin{figure}
	\centering
	\includegraphics[scale=0.4]{../Programacion para ML/Contenido/img/BD+ML+CC.png}
\end{figure}
\end{frame}

\begin{frame}{Diseño de un sistema de aprendizaje}
\begin{itemize}\pause
	\item Cuando se aplica ML a un conjunto de datos se tiene un proyecto\pause
	\item Un proyecto de ML no es lineal pero considera al menos lo siguiente\pause
\end{itemize}

\begin{enumerate}
	\item Definir el problema\pause
	\item Preparar los Datos\pause
	\item Evaluar algoritmos\pause
	\item Presentar resultados
\end{enumerate}
\end{frame}

\begin{frame}{}
\begin{figure}[H]
	\centering
	\includegraphics[scale=0.3]{../Programacion para ML/Contenido/img/procesoML1.png}
\end{figure}
\end{frame}

\begin{frame}{}
\begin{figure}[H]
	\centering
	\includegraphics[scale=0.4]{../Programacion para ML/Contenido/img/procesoML2.jpg}
\end{figure}
\end{frame}

\begin{frame}{}
\begin{figure}[H]
	\centering
	\includegraphics[scale=0.6]{../Programacion para ML/Contenido/img/procesoML3.png}
\end{figure}
\end{frame}

\section{Consideraciones éticas}
\begin{frame}{Consideraciones éticas}
\begin{itemize}
	\item La práctica de la ética ayuda a los profesionales a dar un paso atrás y evaluar una situación desde el punto de vista ético. \pause
	\item Principalmente, la ética de datos está diseñada para ayudarnos a detenernos a pensar con la finalidad de comprender cómo 
		enfrentar dilemas tanto a nivel personal 
como profesional.
\end{itemize}
\end{frame}

\begin{frame}{Privacidad}
\begin{itemize}
	\item La privacidad es un derecho humano intrínseco, que muchas personas desconocen, no ejercen o incluso renuncian a el.\pause
	\item Sin embargo, la privacidad no es un tema menor y en la actualidad debe preocuparnos por lo menos conocer ese derecho.\pause
	\item Conforme el Internet de las cosas crezca, será cada vez más difícil para los usuarios mantener su privacidad. 
	\item Los dispositivos que utilizamos todos los días tienen muchos sensores que capturan en todo momento datos relacionados a 
		nosotros y nuestra actividad. \pause
	\item Si un usuario no conoce la política de uso de estos datos por parte del fabricante, concede libertad completa del uso de 
		dichos datos.\pause
	\item Por otro lado, si un usuario desea leer la política de uso de datos de algún fabricante, esta no será fácilmente comprensible 
		debido a los tecnicismos que un documento de esta naturaleza posee. \pause
	\item En muchas ocasiones no sólo se trata de desconocimiento o apatía del usuario, puede ser el caso de que un fabricante o 
		proveedor no le facilite la tarea al usuario. 
\end{itemize}
\end{frame}

\begin{frame}{Privacidad}
\begin{block}{Privacidad}
El principal obstáculo de la privacidad consiste en que los consumidores desconozcan que se estén recolectando datos acerca de el sin su consentimiento.
\end{block}
\end{frame}

\begin{frame}{}
\begin{figure}[H]
	\centering
	\includegraphics[scale=0.6]{../Programacion para ML/Contenido/img/ads.jpg}
\end{figure}
\end{frame}

\begin{frame}{Caso Target}
\begin{itemize}
	\item En 2012 Target recibió prensa negativa por identificar a mujeres embarazadas.\pause
	\item Cuando Charles Duhigg (New York Times) investigaba sobre el valor de los datos en las empresas.\pause
	\item Comenzó con la queja de un padre de Mineapolis al gerente de la tienda cuando su hija recibió cupones de descuento  en ropa de bebés y cunas.\pause
	\item Pensó que los cupones eran inadecuadas y promovían el embarazo adolescente, pero resultó que su hija  si estaba embarazada.\pause
\end{itemize}
\end{frame}

\begin{frame}{Caso Target}
\begin{itemize}
	\item Target uso modelos predictivos.\pause
	\item Se centró en las mujeres que se habían inscrito en mesas de regalo de bebés.\pause
	\item Compararon el comportamiento de compra de esas mujeres con el comportamiento de los demás clientes.\pause
	\item Encontraron 25 variables, entre ellas:  gran cantidad de lociones sin perfume; suplementos como calcio, magnesio y zinc; jabones libre de fragancia; 
		bolsas de algodón jumbo y desinfectante para manos.
\end{itemize}
\end{frame}

\begin{frame}{Caso Target}
\begin{figure}[H]
	\centering
	\includegraphics[scale=0.3]{../Programacion para ML/Contenido/img/target.png}
\end{figure}
\end{frame}

\end{document}
